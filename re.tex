
Ching-Hua Chang
E-mail: sp0500@gmail.com
Phone: +886-937-601820
Home Address
8F., No.403, Sec. 2, Guangfu Rd., East Dist., Hsinchu City 300, Taiwan (R.O.C.)




PROFESSIONAL SKILLS
10+ years of 32/64 bit RISC(ARM/S+core hardware/software co-design and verification experience.
10+ years of Linux Kernel Porting in ARM, MIPS and S+core based SOC and Device Driver Development experience.
10+ years of software development (C/C++ compiler, assembler, linker, debugger, instruction set simulator, tiny OS kernel, libraries, tools, etc. ) under Linux using C, shell script, etc.
6+ years of 32bit RISC ISA profiling and design experience from scratch
6+ years of 32bit RISC hardware/software co-design and verification experience.
6+ years of Peripheral IP verification experience



Major Projects and Key Achievements

S+core7

ISA survey and define for Sunplus 32 bit RISC S+core7:
Analyze the Instruction Set Architecture (ISA) of ARM and MIPS to define the proprietary ISA of S+core7.
A 16-32 bit hybrid architecture, similar to ARM Thumb2, is employed to reduce code size and enhance performance. This represents a key feature of the S+core ISA.


Hardware/Software Co-designed for In-house 32bit RISC S+core7:
While hardware designers are working on the S+core RTL, we develop a co-verification environment utilizing the Instruction Set Simulator.
This environment is connected to the RTL through the PLI interface, enabling a comparison of instruction behaviors on the hardware RTL simulator.
Eventually, this setup transforms into a random test pattern generator, facilitating randomized testing between hardware and software components.

System software for In-house 32bit RISC S+core7[1]:
Following the definition of the S+core ISA, I undertook the responsibility of leading a team of five colleagues in the development of system software for the S+core architecture. This comprehensive task involved creating a C/C++ compiler, assembler, linker, debugger, instruction set simulator, a streamlined OS kernel, and associated libraries.
The successful adaptation of the GNU toolchain to the S+core architecture was not only achieved, but we also took pride in contributing our achievements to the open-source community.

S+core3(3 pipeline stages) hardware/software co-design and verification:

In house ARMV7/V8 CPU system level verification.

Hardware/software co-verification.

Linux kernel, device driver porting and maintain for DTV, STB SOC


S+core:
https://www.softwaredriverdownload.com/sunplus_score_linux_drivers.html
https://w3.sunplus.com/tw/press/press.asp?id=30291837281A242



SPG290:
https://www.sunplus.com/press/press.asp?id=10338C4323417
https://www.softwaredriverdownload.com/sunplus_score_linux_drivers.html

SPV7050:
https://www.sunplus.com/products/spv7050.asp

SPIF250A/260A:
https://www.sunplusit.com/EN/News/Content/2B24123223156
https://www.sunplus.com/press/press.asp?id=1B3D1174A15291
